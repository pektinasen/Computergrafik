\documentclass[a4paper]{scrartcl}
\usepackage[utf8]{inputenc}  
\usepackage[T1]{fontenc}         
\usepackage[ngerman]{babel}

\usepackage{graphicx} % For includegraphics.
\usepackage{listings} % For code listings.

\usepackage{amsmath} % For easier usage of matrices and stuff
\usepackage{amsfonts}
\usepackage{gauss}

\usepackage[
pdftitle={Computergrafik: Übung 01},
pdfsubject={Abgabe zu Übung 01},
pdfauthor={Sascha Gennrich (4301150), Malte Rohde (4287463)}
]{hyperref}

\title{Computergrafik: Übung 01}
\author{Sascha Gennrich (4301150) \and Malte Rohde (4287463)}
\date{\today{}}

\begin{document}
\maketitle

\section*{Aufgabe 1}
\subsection*{(a)}
Die projektive Transformation besteht aus einer Rotation und einer Translation. Die Translation dient der Verschiebung des Ursprungs $\left(\begin{matrix} 0 \\ 0 \end{matrix}\right)$ auf den Punkt $\left(\begin{matrix} 2 \\ 1 \end{matrix}\right)$. Die Rotation muss folglich den Punkt $\left(\begin{matrix} 3 \\ 4 \end{matrix}\right)$ auf den Punkt $\left(\begin{matrix} 2 \\ 6 \end{matrix}\right) - \left(\begin{matrix} 2 \\ 1 \end{matrix}\right) = \left(\begin{matrix} 0 \\ 5 \end{matrix}\right)$ abbilden. Wir bestimmen den Rotationswinkel $\alpha$ als den zwischen dem Vektor $\left(\begin{matrix} 3 \\ 4 \end{matrix}\right)$ und der y-Achse eingeschlossenen Winkel.

\begin{align*}
\tan\alpha& = \frac{3}{4} \\
\alpha& = \arctan{\frac{3}{4}} \\
& = 36,87^\circ
\end{align*}

Die dritte Spalte der Rotationsmatrix entspricht der Translationsbewegung. Die oberen linken 4 Komponenten ergeben sich aus der kartesischen Rotationsmatrix. Es ergibt sich also: 

\begin{align*}
\left(\begin{matrix} m_{11} & m_{12} & m_{13} \\ m_{21} & m_{22} & m_{23} \\ 0 & 0 & 1 \end{matrix}\right) &
= \left(\begin{matrix} \cos\alpha & \text{-}\sin\alpha & 2 \\ \sin\alpha & \cos\alpha & 1 \\ 0 & 0 & 1 \end{matrix}\right) \\
& = \left(\begin{matrix} \frac{4}{5} & \text{-}\frac{3}{5} & 2 \\ \frac{3}{5} & \frac{4}{5} & 1 \\ 0 & 0 & 1 \end{matrix}\right)
\end{align*}

\subsection*{(b)}
\begin{align*}
\vec{z} & = R\left(\vec{z}\right) \\
\left(\begin{matrix} z_1 \\ z_2 \\ 1 \end{matrix}\right) & = 
\left(\begin{matrix} \frac{4}{5} & \text{-}\frac{3}{5} & 2 \\ \frac{3}{5} & \frac{4}{5} & 1 \\ 0 & 0 & 1 \end{matrix}\right)
\cdot
\left(\begin{matrix} z_1 \\ z_2 \\ 1 \end{matrix}\right)
\end{align*}
Matrix-Vektor-Multiplikation und Lösung des linearen Gleichungssystems:
\begin{align*}
z_1 & = \frac{4}{5} z_1 - \frac{3}{5} z_2 + 2 \\
z_2 & = \frac{3}{5} z_1 + \frac{4}{5} z_2 + 1 \\
\\
\frac{1}{5} z_1 & =  \text{-}\frac{3}{5} z_2 + 2 \\
\frac{1}{5} z_2 & = \frac{3}{5} z_1 + 1 \\
\\
z_1 & =  \text{-} 3 z_2 + 10 \\
z_2 & = 3 z_1 + 5 \\
\\
z_1 & = \text{-}9 z_1 - 15 + 10 \\
    & = \text{-}\frac{1}{2} \\
\\
z_2 & = \text{-}\frac{3}{2} + 5 \\
    & = \frac{7}{2}
\end{align*}

\subsection*{(c)}
Wir berechnen den Drehwinkel über den zwischen den Richtungsvektoren vom Fixpunkt zum Ursprung und vom Fixpunkt zum Punkt $\left(\begin{matrix} 2 \\ 1 \end{matrix}\right)$ eingeschlossenen Winkel. Die Drehrichtung ist in dieser Betrachtungsweise gegen den Uhrzeigersinn.
\begin{align*}
\cos\phi & = \frac{\vec{v_1}\cdot\vec{v_2}}{\left|\vec{v_1}\right|\cdot\left|\vec{v_1}\right|} \\
\text{mit} \\
\vec{v_1} & = \left(\begin{matrix} 0 \\ 0 \end{matrix}\right)
- \left(\begin{matrix} \text{-}\frac{1}{2} \\ \frac{7}{2} \end{matrix}\right) 
= \left(\begin{matrix} \frac{1}{2} \\ \text{-}\frac{7}{2} \end{matrix}\right) \\
\vec{v_2} & = \left(\begin{matrix} 2 \\ 1 \end{matrix}\right)
- \left(\begin{matrix} \text{-}\frac{1}{2} \\ \text{-}\frac{5}{2} \end{matrix}\right) 
= \left(\begin{matrix} \frac{5}{2} \\ \text{-}\frac{5}{2} \end{matrix}\right) \\
\\
\cos\phi & = \frac{\frac{1}{2}\cdot\frac{5}{2} + \text{-}\frac{7}{2} \cdot \text{-}\frac{5}{2}}
{\sqrt{\frac{1}{2}^2 + \left(\text{-}\frac{7}{2}\right)^2} \cdot \sqrt{\frac{5}{2}^2 + \left(\text{-}\frac{5}{2}\right)^2}} \\
 & = \frac{\frac{40}{4}}{\frac{50}{4}} \\
 & = \frac{4}{5} \\
\Longrightarrow \phi & = 36,87^\circ
\end{align*}


\section*{Aufgabe 7: Rotation um eine beliebige Achse}
Wir folgen der in der Aufgabenstellung gegebenen Lösungsidee. Gesucht sind also die Matrizen $A$ (Drehung von $u$ in die $yz$-Ebene), $B$ (Drehung auf die $z$-Achse) und $C$ (Rotation um die $z$-Achse) sowie die Inversen $A^{\text{-}1}$ und $B^{\text{-}1}$.

\subsection*{Drehung von $u$ in die $yz$-Ebene}
Wir drehen Vektor $u=\begin{pmatrix} u_x \\ u_y \\ u_z \end{pmatrix}$ um die $z$-Achse in die $yz$-Ebene. Der Drehwinkel $\alpha$ entspricht dem zwischen dem Vektor $u$ und der $yz$-Ebene eingeschlossenen Winkel. Wir errechnen diesen über Dreiecksgeometrie:
\begin{align*}
\sin\alpha & = \frac{u_x}{\left|u\right|_{xy}} \\
\cos\alpha & = \frac{u_y}{\left|u\right|_{xy}}
\end{align*}
$\left|u\right|_{xy}$ steht hierbei für die Länge des auf die $xy$-Ebene projizierten Vektors, bzw. die Länge von $u$ ohne Berücksichtigung der $z$-Komponente. Die Rotationsmatrix ist entsprechend:
\begin{align*}
A & = \begin{pmatrix} \cos\alpha & \text{-}\sin\alpha & 0 \\ \sin\alpha & \cos\alpha & 0 \\ 0 & 0 & 1 \end{pmatrix}
\\
& = \begin{pmatrix} 
\frac{u_y}{\left|u\right|_{xy}} & \text{-} \frac{u_x}{\left|u\right|_{xy}} & 0 \\ 
\frac{u_x}{\left|u\right|_{xy}} & \frac{u_y}{\left|u\right|_{xy}} & 0 \\ 
0 & 0 & 1
\end{pmatrix}
\end{align*}
Mit Hilfe der Rotationsmatrix wird der Vektor $u$ auf den Vektor $v$ abgebildet:
\begin{align*}
\vec{v} & = A\vec{u} \\
& = \begin{pmatrix} 0 \\ \frac{u_x^2 + u_y^2}{\left|u\right|_{xy}} \\ u_z\end{pmatrix}
\end{align*}

\subsection*{Berechnung der Inversen $A^{\text{-}1}$ für später}
Wie berechnen die Inverse $A^{\text{-}1}$ mit Hilfe des Gauß-Jordan-Algorithmus.
\begin{align*}
\begin{gmatrix}[p]
\frac{u_y}{\left|u\right|_{xy}} & \text{-} \frac{u_x}{\left|u\right|_{xy}} & 0 \\ 
\frac{u_x}{\left|u\right|_{xy}} & \frac{u_y}{\left|u\right|_{xy}} & 0 \\ 
0 & 0 & 1
\rowops
\mult{0}{\cdot \text{-}\frac{u_x}{u_y}}
\add{0}{1}
\mult{1}{\cdot \text{-}\frac{u_x^2}{u_x^2 + u_y^2}}
\add{1}{0}
\mult{0}{\cdot \text{-}\frac{\left|u\right|_{xy}}{u_x}}
\mult{1}{\cdot \text{-}\frac{\left|u\right|_{xy} \cdot u_y}{u_x^2}}
\end{gmatrix}
\\
\Longrightarrow A^{\text{-}1} = 
\begin{pmatrix}
\frac{u_y \cdot \left|u\right|_{xy}}{u_x^2 + u_y^2} & \frac{u_x \cdot \left|u\right|_{xy}}{u_x^2 + u_y^2} & 0 \\
\text{-}\frac{u_x \cdot \left|u\right|_{xy}}{u_x^2 + u_y^2} & \frac{u_y \cdot \left|u\right|_{xy}}{u_x^2 + u_y^2} & 0 \\
0 & 0 & 1
\end{pmatrix}
\end{align*}


\subsection*{Drehung um die $x$-Achse auf die $z$-Achse}
Um Vektor $v$ auf die $z$-Achse abzubilden, rotieren wir (in der $yz$-Ebene) um die $x$-Achse. Den Drehwinkel $\beta$ erhalten wir wie zuvor über Dreiecksgeometrie:
\begin{align*}
\sin\beta & = \frac{v_y}{\left|v\right|} \\
\cos\beta & = \frac{v_z}{\left|v\right|}
\end{align*}
Rotationsmatrix B:
\begin{align*}
B & = \begin{pmatrix} 1 & 0 & 0 \\ 0 & \cos\beta & \text{-}\sin\beta\\ 0 & \sin\beta & \cos\beta \end{pmatrix}
\\
& = \begin{pmatrix} 1 & 0 & 0 \\ 0 & \frac{v_z}{\left|v\right|} & \text{-}\frac{v_y}{\left|v\right|} \\ 0 & \frac{v_y}{\left|v\right|} & \frac{v_z}{\left|v\right|} \end{pmatrix}
\end{align*}

\subsection*{Berechnung der Inversen $B^{\text{-}1}$ für später}
Analog zu oben:
\begin{align*}
B^{\text{-}1} = 
\begin{pmatrix}
0 & 0 & 1 \\
0 & \frac{v_z \cdot \left|v\right|}{v_y^2 + v_z^2} & \frac{v_y \cdot \left|v\right|}{v_y^2 + v_z^2}\\
0 & \text{-}\frac{v_y \cdot \left|v\right|}{v_y^2 + v_z^2} & \frac{v_z \cdot \left|v\right|}{v_y^2 + v_z^2}
\end{pmatrix}
\end{align*}

\subsection*{Drehung um die $z$-Achse}
Wir rotieren um die $z$-Achse mit $\phi = 30^\circ$. Rotationsmatrix C:
\begin{align*}
A & = \begin{pmatrix} \cos\phi & \text{-}\sin\phi & 0 \\ \sin\phi & \cos\phi & 0 \\ 0 & 0 & 1 \end{pmatrix}
\\
& = \begin{pmatrix} \frac{\sqrt{3}}{2} & \text{-}\frac{1}{2} & 0 \\ \frac{1}{2} & \frac{\sqrt{3}}{2} & 0 \\ 0 & 0 & 1\end{pmatrix}
\end{align*}

\subsection*{Affine Transformation}
Die gesamte Transformation ergibt sich als:
\begin{equation*}
T = ABCA^{\text{-}1}B^{\text{-}1}
\end{equation*}
\end{document}
